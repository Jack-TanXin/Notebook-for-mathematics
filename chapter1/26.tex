\section{极值点的判定}

\begin{theorem}
	设$f$在$x_0$连续,在某邻域$U^o(x_0,\delta)$上可导,若当$x\in(x_0-\delta,x_0)$时$f'(x)\leq(\geq) 0$.若当$x\in (x_0,x_0+\delta)$时$f'(x)\geq(\leq) 0$,则$f(x)$在$x_0$处取极小(大)值.
\end{theorem}

这是最常见的一种极值点判定的方法,还有另一种在一些情况有奇效:

\begin{theorem}
	设$f$在$x_0$的某邻域$U(x_0,\delta)$上存在$n-1$阶导数,在$x=x_0$处存在$n$阶导数,且$f^{(k)}(x_0)=0(k=1,2,\dots,n-1),f^{(n)}\neq 0$,则有:
	\begin{enumerate}
		\item 当$n$为偶数,$f$在$x_0$处取极值.且当$f^{(n)}(x_0)<0$时取极大值,当$f^{(n)}(x_0)>0$时取极小值.
		\item 当$n$为奇数,$f$在$x_0$处不取极值.
	\end{enumerate}
\end{theorem}

注意:这两种方法都是充分条件,反之不一定成立.
