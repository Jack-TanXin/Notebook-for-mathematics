\section{一般函数反常积分的敛散判别}

\begin{theorem}[Dirichlet判别法]
	若$F(u) = \int_{a}^{u}f(x)\textbf{d}x$在$[a,+\infty)$上有界,$g(x)$在$[a,+\infty)$上当$x\to +\infty$时单调趋于0,则$\int_{a}^{+\infty}f(x)g(x)\textbf{d}x$收敛.
\end{theorem}

\begin{theorem}[Albel判别法]
	若$\int_{a}^{+\infty}f(x)\textbf{d}x$收敛,$g(x)$在$[a,+\infty)$上单调有界,则$\int_{a}^{+\infty}f(x)g(x)\textbf{d}x$收敛.
\end{theorem}

上面两个定理是针对无穷反常积分的,下面是瑕积分的对应形式:

\begin{theorem}[Dirichlet判别法]
	设$a$为$f(x)$的瑕点,若$F(u) = \int_{u}^{b}f(x)\textbf{d}x$在$(a,b]$上有界,$g(x)$在$(a,b]$上单调且$\displaystyle\lim_{x\to a^+}g(x)=0$,则瑕积分$\int_{a}^{+\infty}f(x)g(x)\textbf{d}x$收敛.
\end{theorem}

\begin{theorem}[Albel判别法]
	设$a$为$f(x)$的瑕点,若$\int_{a}^{b}f(x)\textbf{d}x$收敛,$g(x)$在$(a,b]$上单调有界,则$\int_{a}^{+\infty}f(x)g(x)\textbf{d}x$收敛.
\end{theorem}