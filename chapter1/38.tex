\section{定积分与周期函数}

\begin{proposition}[Riemann-Lebesgue引理]
	设$f(x)$在$[a,b]$上可积,$g(x)$以$T$为周期,且在$[0,T]$上可积,则有:
	\begin{equation}
		\displaystyle\lim_{n\to\infty}\int_{a}^{b}f(x)g(nx)\textbf{d}x = \dfrac{1}{T}\int{0}^{T}g(x)\textbf{d}x\int_{a}^{b}f(x)\textbf{d}x.
	\end{equation}
\end{proposition}



\begin{proposition}[Riemann-Lebesgue引理]
	设$f(x)$在任意有限区间可积,且$\int_{a}^{+\infty}f(x)\textbf{d}x$绝对收敛,$g(x)$是以$T$为周期的可积函数,则有:
	\begin{equation}
			\displaystyle\lim_{n\to\infty}\int_{a}^{+\infty}f(x)g(nx)\textbf{d}x = \dfrac{1}{T}\int{0}^{T}g(x)\textbf{d}x\int_{a}^{+\infty}f(x)\textbf{d}x.
	\end{equation}
\end{proposition}


这个引理在三角函数中的应用是非常重要的!

\begin{corollary}
	设$f(x)$在$[a,b]$($b$可以是$+\infty$)上可积且绝对可积,则有:
	\begin{equation}
		\displaystyle\lim_{\lambda\to +\infty}\int_{a}^{b}f(x)\cos{\lambda x}\textbf{d}x = 0.
	\end{equation}
	\begin{equation}
		\displaystyle\lim_{\lambda\to +\infty}\int_{a}^{b}f(x)\sin{\lambda x}\textbf{d}x = 0.
	\end{equation}
\end{corollary}