\section{一个易错的命题}

\begin{proposition}
	对任意的$p\in N^+$,都有
	\[\displaystyle\lim_{n\to\infty}(a_{n+p}-a_n)=0,\]
	但数列$\{a_n\}$未必收敛.
\end{proposition}

\begin{example}
	我们在这里可以给出反例来说明本命题的正确:设$a_n=1+\dfrac{1}{2}+\dfrac{1}{3}+\dots+\dfrac{1}{n}.$由柯西收敛准则知:$\{a_n\}$是发散数列.又对$\forall p\in N^+$有:
	\[a_{n+p}-a_n=\dfrac{1}{n+1}+\dfrac{1}{n+2}+\dots+\dfrac{1}{n+p}\leq\dfrac{p}{n+1}\rightarrow 0,n\rightarrow\infty.\]
	故命题证毕.
\end{example}

接下来补充一个命题,这是华东师大课本上的一道课后习题,大家可以比较一下这两道题目,同时思考一下二者的关系,以及命题1.1.9中原数列不收敛的原因

\begin{proposition}
	已知$\displaystyle\lim_{n\to\infty}(a_n)=a$,则对任意的$k\in N^+$,都有
	\begin{equation}
		\displaystyle\lim_{n\to\infty}a_{n+k}=a.
	\end{equation}
\end{proposition}