\section{求极限时变量替换的陷阱}

\begin{proposition}[变量替换的充分条件]
	设$\displaystyle\lim_{x\to a}g(x)=A$,$\displaystyle\lim_{y\to A}f(x)=B$成立,且在点$a$的某个领域上有$g(x)=y$,如果满足下面三个条件之一:
	\begin{enumerate}
		\item 存在点$a$的一个空心领域$O_{\delta}\{a\}-\{a\}$,在其中$g(x)\neq A$.
		\item $\displaystyle\lim_{y\to A}f(y)=f(A)$.
		\item $A=\infty$且$\displaystyle\lim_{y\to A}f(y)存在$
	\end{enumerate}	
	则有:
	\begin{equation}
		\displaystyle\lim_{x\to a}f(g(x))=\displaystyle\lim_{y\to A}f(y)=B.
	\end{equation}
\end{proposition}

注意:在使用变量替换求函数极限的时候要注意条件. 具体来说就是在求极限$\displaystyle\lim_{x\to a}F(a)$时,如果有$F(x)=f(g(x))$,又有$\displaystyle\lim_{x\to a}g(x)=A,\displaystyle\lim_{y\to A}f(y)=B$,能否推出$\displaystyle\lim_{x\to A}F(x)=B?$

\textbf{不能!}可以给出反例:设函数$f(x)$和$g(x)$分别为:
\begin{equation}
	f(y)=\left\{
	\begin{aligned}
		1 \quad y=0\\
		0 \quad y\neq 0\\
	\end{aligned}
	\right
	.
\end{equation}
\begin{equation}
	g(x)\equiv 0.
\end{equation}
则有$f(g(x))\equiv 1$,但$\displaystyle\lim_{y\to 0}f(y)=0,\displaystyle\lim_{x\to 0}f(g(x))=1.$
