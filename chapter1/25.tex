\section{两类詹森不等式}

詹森不等式是凸函数的重要性质之一类似于柯西不等式的,詹森不等式也有离散和连续两个版本,且表现为求和和积分形式。
\begin{theorem}[离散的詹森不等式]
	设函数$f(x)$为$[a,b]$上的凸函数,则对$\forall x_i\in[a,b],\lambda_i>0(i=1,2,\dots,n)$,有:
	\begin{equation}
		f\left(\dfrac{1}{\displaystyle\sum_{i=1}^{n}\lambda_i}\displaystyle\sum_{i=1}^{n}\lambda_ix_i\right)\leq\dfrac{\displaystyle\sum_{i=1}^{n}\lambda_if(x_i)}{\displaystyle\sum_{i=1}^{n}\lambda_i}.
	\end{equation}
\end{theorem}
	
\begin{theorem}[连续的詹森不等式]
	设函数$f(x)$和$g(x)$在$[a,b]$上连续,其中$g(x)>0,m\leq f(x)\leq M$,如果$\varphi(x)$是$[m,M]$上的连续的凸函数,则有:
	\begin{equation}
		f\left(\dfrac{\int_{a}^{b}f(x)g(x)\textbf{d}x}{\int_{a}^{n}g(x)\textbf{d}x}\right)\leq\dfrac{\int_{a}^{b}\varphi(f(x))g(x)\textbf{d}x}{\int_{a}^{b}g(x)\textbf{d}x}.
	\end{equation}
\end{theorem}