\section{三大微分中值定理}

\begin{theorem}[罗尔中值定理]
	若函数$f$同时满足如下三个条件:
	\begin{enumerate}
		\item 在闭区间$[a,b]$连续;
		\item 在开区间$(a,b)$可导;
		\item $f(a)=f(b)$.
	\end{enumerate}
	则在$(a,b)$上至少存在一点$\zeta$使得:
	\begin{equation}
		f'(\zeta)=0.
	\end{equation}
\end{theorem}

\begin{theorem}[拉格朗日中值定理]
	若函数$f$同时满足如下两个条件:
	\begin{enumerate}
		\item 在闭区间$[a,b]$连续;
		\item 在开区间$(a,b)$可导;
	\end{enumerate}
	则在$(a,b)$上至少存在一点$\zeta$使得:
	\begin{equation}
		f'(\zeta)=\dfrac{f(b)-f(a)}{b-a}.
	\end{equation}
\end{theorem}

\begin{theorem}[柯西中值定理]
		若函数$f$与$g$同时满足如下四个条件:
	\begin{enumerate}
		\item 在闭区间$[a,b]$连续;
		\item 在开区间$(a,b)$可导;
		\item $\forall x\in [a,b]$有$f'(x)\neq g'(x))$.
		\item $g(a)\neq g(b)$.
	\end{enumerate}
	则在$(a,b)$上至少存在一点$\zeta$使得:
	\begin{equation}
		\dfrac{f'(\zeta)}{g'(\zeta)}=\dfrac{f(b)-f(a)}{g(b)-g(a)}.
	\end{equation}
	\end{theorem}