\section{实数系的基本定理}

下面的定理为实数系基本定理,以任意的一个为已知条件可以推出另外五个,请务必注意理解这六个定理:

\begin{theorem}[确界存在定理]
	在实数集中,有上界的数集一定有上确界,有下界的数集一定有下确界.
\end{theorem}

\begin{theorem}[单调有界定理]
	单调有界数列一定收敛.
\end{theorem}

\begin{theorem}[Cauchy收敛准则]
	数列$\{a_n\}$收敛的充要条件是:对于$\forall\epsilon>0$,存在$N\in N^+$,使得$\forall m,n>N$都有不等式:
	\begin{equation}
		|a_m-a_n|<\epsilon
	\end{equation}
	成立.
\end{theorem}

\begin{theorem}[闭区间套定理]
	设有闭区间序列$\{[a_n,b_n]\}$满足条件:
	\begin{equation}
		a_n\leq a_{n+1}\leq b_{n+1}\leq b_n,n\in N^+,
	\end{equation}
	则存在$\zeta$使得$a_n\leq\zeta\leq b_n,n\in N^+$.若有:
	\begin{equation}
		\displaystyle\lim_{n\to\infty}|b_n-a_n|=0,
	\end{equation}
	则$\zeta$唯一,且数列$\{a_n\}$与$\{b_n\}$从$\zeta$的两侧单调收敛于$\zeta$.
\end{theorem}

\begin{theorem}[聚点定理]
	有界数列必有收敛子集.
\end{theorem}

\begin{theorem}[有限覆盖定理]
	设$[a,b]\subset\displaystyle\bigcup_{\alpha}O_{\alpha}$,其中每个$O_{\alpha}$都是开区间(此时我们称$O_{\alpha}$为$[a,b]$上的开覆盖),则存在$\{O_{\alpha}\}$的有限子集$\{O_1,O_2,\dots,O_n\}$是区间的开覆盖,即:
	\begin{equation}
		[a,b]\subset\displaystyle\bigcup_{i=1}^{n}O_{i}.
	\end{equation}
\end{theorem}

这6条定理给我背得死死的,彻底理解清楚!