\section{积分中值定理}

\begin{theorem}[积分第一中值定理]
	若$f$为$[a,b]$上连续,则至少存在一点$\xi\in[a,b]$,使得:
	\begin{equation}
		\int_{a}^{b} f(x)\textbf{d}x = f(\xi)(b-a).
	\end{equation}
\end{theorem}

\begin{corollary}[推广的积分第一中值定理]
	若$f$与$g$都在$[a,b]$上连续,且$g$在$[a,b]$上不变号,则至少存在一点$\xi\in[a,b]$,使得:
	\begin{equation}
		\int_{a}^{b}f(x)g(x)\textbf{d}x = f(\xi)\int_{a}^{b}g(x)\textbf{d}x.
	\end{equation}
\end{corollary}

\begin{theorem}[积分第二中值定理]
	若函数$f$在$[a,b]$上可积:
	\begin{enumerate}
		\item 若函数$g$在$[a,b]$上单调递减,且$g(x)\geq 0$,则存在$\xi\in[a,b]$,使得:
		\begin{equation}
			\int_{a}^{b}f(x)g(x)\textbf{d}x = g(a)\int_{a}^{\xi} f(x)\textbf{d}x.
		\end{equation}
		\item 若函数$g$在$[a,b]$上单调递增,且$g(x)\geq 0$,则存在$\eta\in[a,b]$,使得:
		\begin{equation}
			\int_{a}^{b}f(x)g(x)\textbf{d}x = g(b)\int_{\eta}^{b} f(x)\textbf{d}x.
		\end{equation}
	\end{enumerate}
\end{theorem}