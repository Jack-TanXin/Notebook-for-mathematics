\section{定积分的应用}

\begin{proposition}[平面图形的面积]
	由上下两条连续曲线$y=f_1(x)$和$y = f_2(x)$以及两条直线$x=a$与$x=b(a<b)$所围成的平面图形的面积为:
	\begin{equation}
		A = \int_{a}^{b}f_2(x)-f_1(x)\textbf{d}x.
	\end{equation}
\end{proposition}


\begin{proposition}[极坐标下的面积]
	设曲线$C$是由极坐标方程
	\begin{equation}
		r = r(\theta),\theta\in[\alpha,\beta]
	\end{equation}
	给出,其中$r(\theta)$在$[\alpha,\beta]$上连续,$\beta-\alpha\leq 2\pi$.由曲线$C$与两射线$\theta=\alpha,\theta=\beta$所围成的平面图形(也成为扇形)的面积为:
	\begin{equation}
		A = \dfrac{1}{2}\int_{\alpha}^{\beta}r^2(\theta)\textbf{d}\theta.
	\end{equation}
\end{proposition}


\begin{proposition}[旋转体的截面积与体积]
	设$f$是$[a,b]$上的连续函数,$\Omega$是由平面图形
	\begin{equation}
		0\leq |y|\leq |f(x)|,a\leq x\leq b.
	\end{equation}
	绕$x$轴旋转一周所得的旋转体,那么易知截面面积函数为:
	\begin{equation}
		A(x)=\pi f^2(x),x\in[a,b].
	\end{equation}
	而旋转体$\Omega$的体积公式为:
	\begin{equation}
		V = \pi \int_{a}^{b}f^2(x)\textbf{d}x.
	\end{equation}
\end{proposition}


\begin{proposition}[弧长与曲率]
	设曲线$C$由参数方程
	\begin{equation}
		\left\{
		\begin{aligned}
			x = x(t) \\
			y = y(t) \\
		\end{aligned}
		\right
		.
	\end{equation}
	给出.若曲线$C$是一条光滑曲线,则$C$是可求长的,且弧长为:
	\begin{equation}
		s = \int_{\alpha}^{\beta}\sqrt{(x'(t)^2)+(y'(t))^2}\textbf{d}t.
	\end{equation}
	曲率计算公式为:
	\begin{equation}
		K = \dfrac{|x'y''-x''y'|}{\left( (x')^2+(y')^2  \right)^{\frac{3}{2}}}
	\end{equation}
\end{proposition}


\begin{corollary}[极坐标方程的弧长]
	若曲线$C$是由极坐标方程
	\begin{equation}
		r = r(\theta),\theta\in[\alpha,\beta]
	\end{equation}	
	表示,则弧长公式为:
	\begin{equation}
		s = \int_{\alpha}^{\beta}\sqrt{r^2(\theta)+\left(r'(\theta)\right)^2}\textbf{d}\theta.
	\end{equation}
\end{corollary}





