\section{可微性、连续性与区间的渊源}

\begin{definition}[单点连续]
	设$f$为定义在$U(x_0)$上的函数,若对$x_0\in U(x_0)$有:
	\begin{equation}
		\displaystyle\lim_{x\to x_0}f(x)=f(x_0).
	\end{equation}
	则称$f$在$x_0$处连续.
\end{definition}

\begin{definition}[单点可微]
	设$f$为定义在$U(x_0)$上的函数,若对$x_0\in U(x_0)$有:
	\begin{equation}
		\displaystyle\lim_{x\to x_o}\dfrac{f(x)-f(x_0)}{x-x_0}
	\end{equation}
	存在,则称$f$在$x_0$处可导.
\end{definition}

由此可见:想要函数$f$在$x_0$连续或可导,那么$f$必须在$U(x_0)$有定义.

\begin{theorem}
	若$f$在$x_0$上可导,则$f$在$x_0$上连续.
\end{theorem}

\begin{definition}[单点二阶可导]
	设$f$的导函数$f'(x)$在点$x_0$可导,即:
	\begin{equation}
		\displaystyle\lim_{x\to x_o}\dfrac{f'(x)-f'(x_0)}{x-x_0}
	\end{equation}
	存在,则将其记为$f''(x_0)$,并称$f$在$x_0$处二阶可导.
\end{definition}

要搞清楚一点:$f'$要想在$x_0$处可导,则$f'$必须在在$U(x_0)$有定义.

\begin{definition}[补充:连续可微]
	若函数$f$在$I$上可导,且导函数在$I$上连续,则称$f$在$I$上连续可微.
\end{definition}

\begin{proposition}
	\begin{enumerate}
		\item 若函数$f$在区间$I$上可导,则$f$在区间$I$上没有间断点.函数在区间上可导要求函数在区间上不仅连续,而且光滑.且$f'$在I上没有第一类间断点,可能有第二类间断点.
		\item 若函数$f$在$x=x_0$处$n$阶可导,则$f$在$U(x_0)$有$n-1$阶导函数,且$f^{(n-1)}$在$U(x_0)$连续.
		\item 若函数$f$在$U(x_0)$上$n$阶可导,则$f$在$U(x_0)$上$k(k=1,2,\dots,n-1)$阶可导,且$f^{(n)}$在$U(x_0)$连续.
	\end{enumerate}
\end{proposition}