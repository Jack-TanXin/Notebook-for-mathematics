\section{函数的一致收敛}

函数的连续性是函数的局部性质,而一致连续则是函数的全局性质——这是二者最大的区别!!!一定要好好地理解。

\begin{definition}
	设$f(x)$是定义在区间$I$上的函数,若任给的$\epsilon>0$,存在$\delta>0$使得$\forall x',x''\in I(|x'-x''<\delta|)$有不等式:
	\begin{equation}
		|f(x')-f(x'')|<\epsilon
	\end{equation}
	成立,则称$f(x)$在区间$I$上一致连续.
\end{definition}

\begin{theorem}[Cantor定理]
	设$f(x)$为定义在闭区间$[a,b]$上的连续函数,则$f(x)$在闭区间$[a,b]$上一致连续.
\end{theorem}

\begin{corollary}
	设$f(x)$在闭区间$[a,b]$上一致连续,则$f(x)$在闭区间$[c,d]\subset [a,b]$上一致连续.
\end{corollary}

\begin{proposition}
	有界开区间$(a,b)$上的连续函数$f(x)$在$(a,b)$上一致连续的充要条件是:$f(x)$在$x=a$处的右极限和$x=b$处的左极限均存在且有限.
\end{proposition}

\begin{corollary}
	开区间上的一致连续函数一定在开区间上有界.
\end{corollary}

\begin{proposition}
	$f(x)$在$\mathbb{R}$上一致连续,则存在正实数$a,b$使得:
	\begin{equation}
		|f(x)|\leq a|x|+b.
	\end{equation}
\end{proposition}