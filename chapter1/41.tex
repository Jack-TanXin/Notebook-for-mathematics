\section{非负函数反常积分的比较原则}

对于非负函数的两种反常积分的敛散性判别,比较原则是非常好用的方法:

\begin{theorem}[无穷积分比较原则]
	设$f(x),g(x)$为$[a,+\infty)$上的非负函数,它们在任意有限区间$[a,u]$上可积,且有:
	\begin{equation}
		0\leq f(x)\leq g(x),x\in[a,+\infty],
	\end{equation}
	那么当$\int_{a}^{+\infty}g(x)\textbf{d}x$收敛时$\int_{a}^{+\infty}f(x)\textbf{d}x$收敛,当$\int_{a}^{+\infty}f(x)\textbf{d}x$发散时$\int_{a}^{+\infty}g(x)\textbf{d}x$发散.
\end{theorem}


\begin{theorem}[瑕积分比较原则]
	设$f(x),g(x)$为$(a,b]$上以$a$为瑕点的函数,它们在任意有限区间$[u,b]\subset(a,b]$上可积,且有:
	\begin{equation}
		0\leq f(x)\leq g(x),x\in(a,b],
	\end{equation}
	那么当$\int_{a}^{b}g(x)\textbf{d}x$收敛时$\int_{a}^{b}f(x)\textbf{d}x$收敛,当$\int_{a}^{b}f(x)\textbf{d}x$发散时$\int_{a}^{b}g(x)\textbf{d}x$发散.
\end{theorem}


比较原则直接使用并不太方便,一般要使用极限形式下特定函数的比较原则推论进行反常积分敛散性判别:

\begin{corollary}[极限形式下的无穷积分比较原则]
	设$f(x)$是定义于$[a,+\infty)$上的非负函数,在$\forall$有限区间$[a,u]$上可积,且:
	\begin{equation}
		\displaystyle\lim_{x\to +\infty}x^pf(x)=\lambda.
	\end{equation}
	则有:
	\begin{enumerate}
		\item 当$p>1,0\leq\lambda <+\infty$时,$\int_{a}^{+\infty}f(x)\textbf{d}x$收敛;
		\item 当$p\leq 1,0<\lambda\leq +\infty$时,$\int_{a}^{+\infty}f(x)\textbf{d}x$发散;
	\end{enumerate}
\end{corollary}




\begin{corollary}[极限形式下的瑕积分比较原则]
	设$f(x)$是定义于$(a,b]$上的非负函数,$a$为其瑕点,在$\forall$有限区间$[u,b]\subset(a,b]$上可积,且:
	\begin{equation}
		\displaystyle\lim_{x\to +\infty}(x-a)^pf(x)=\lambda.
	\end{equation}
	则有:
	\begin{enumerate}
		\item 当$0<p<1,0\leq\lambda <+\infty$时,$\int_{a}^{b}f(x)\textbf{d}x$收敛;
		\item 当$p\geq 1,0<\lambda\leq +\infty$时,$\int_{a}^{b}f(x)\textbf{d}x$发散;
	\end{enumerate}
\end{corollary}







