\section{典型反常积分、级数敛散性}

\begin{proposition}[$p$积分]
	对于无穷积分$\int_{1}^{+\infty}\dfrac{\textbf{d}x}{x^p}$有:
	\begin{equation}
		\displaystyle\lim_{u \to +\infty}\int_{1}^{u}\dfrac{\textbf{d}x}{x^p} = \left\{
		\begin{aligned}
			\dfrac{1}{p-1} \quad p>1\\
			+\infty \quad p\leq 1\\
		\end{aligned}
		\right
		.
	\end{equation}
	显然其敛散性如下:
	\begin{enumerate}
		\item $p>1$时$\int_{1}^{+\infty}\dfrac{\textbf{d}x}{x^p}$收敛于$\dfrac{1}{p-1}$.
		\item $p\leq 1$时$\int_{1}^{+\infty}\dfrac{\textbf{d}x}{x^p}$发散于$+\infty$.
	\end{enumerate}
\end{proposition}


\begin{proposition}[$q$积分]
	对于瑕积分$\int_{0}^{1}\dfrac{\textbf{d}x}{x^q}$有:
	\begin{equation}
		\int_{u}^{1}\dfrac{\textbf{d}x}{x^q} = \left\{
		\begin{aligned}
			\dfrac{1}{q-1}(1-u^{1-q}) \quad q\neq 1\\
			-\ln{u} \quad q=1\\
		\end{aligned}
		\right
		.
	\end{equation}
	显然其敛散性如下:
	\begin{enumerate}
		\item $0<q<1$时,$\int_{0}^{1}\dfrac{\textbf{d}x}{x^q}$收敛于$\dfrac{1}{1-q}$.
		\item $q\geq 1$时,$\int_{0}^{1}\dfrac{\textbf{d}x}{x^q}$发散于$+\infty$.
	\end{enumerate}
\end{proposition}


\begin{corollary}[$p$级数]
	对于$p$级数$\sum \dfrac{1}{n^p}$的敛散性有:
	\begin{enumerate}
		\item $p>1$时$\sum \dfrac{1}{n^p}$收敛.
		\item $p\leq 1$时$\sum \dfrac{1}{n^p}$发散.
	\end{enumerate}
\end{corollary}

\begin{proposition}[几何级数]
	对于$1+aq+aq^2+\dots+aq^n+\dots$的敛散性,有:
	\begin{equation}
		S_n = 1+aq+aq^2+\dots+aq^n = a \cdot \dfrac{1-q^n}{1-q}.
	\end{equation}
	故有:
	\begin{enumerate}
		\item $|q|<1$时$1+aq+aq^2+\dots+aq^n+\dots$收敛于$\dfrac{a}{1-q}$.
		\item $|q|\geq 1$时$1+aq+aq^2+\dots+aq^n+\dots$发散.
	\end{enumerate}
\end{proposition}


\begin{proposition}[典型级数]
	\begin{enumerate}
		\item 调和级数$1+\dfrac{1}{2}+\dfrac{1}{3}+\dots+\dfrac{1}{n}+\dots$是发散的.
		\item 级数$\sum\dfrac{1}{n^2}$是收敛的.
	\end{enumerate}
\end{proposition}

































