\section{几个很有用的不等式}

\begin{proposition}[Bernoulli不等式]
	设$h>-1,n\in N_+$,则有不等式:
	\begin{equation}
		(1+h)^n\geq 1+nh,
	\end{equation}		
	成立,其中$n>1$时等号成立的充要条件是$h=0$.
\end{proposition}

\begin{corollary}[对Bernoulli不等式的补充]
	设$h>-1,n\in N_+$,则有不等式:
	\begin{equation}
		(1+h)^n\geq 1+\dfrac{n(n-1)h^2}{2},
	\end{equation}
	成立,其中$n>1$时等号成立的充要条件是$h=0$.
\end{corollary}

这个命题以及推论在求数列极限中是有非常大的作用的,均可由二项式公式轻松推得,杀伤力极强!

\begin{proposition}[均值不等式]
	设$a_1,a_2,\dots,a_n$为$n$个非负实数,则有不等式:
	\begin{equation}
		\dfrac{1}{\frac{1}{a_1}+\dots+\frac{1}{a_n}}\leq\sqrt[n]{a_1\dots a_n}\leq\dfrac{a_1+\dots+a_n}{n}\leq\sqrt{\dfrac{{a_1}^2+\dots+{a_n}^2}{n}},
	\end{equation}
	成立,其中等号成立的充要条件时$a_1=a_2=\dots=a_n.$
\end{proposition}

均值不等式链的作用是不言而喻的,在高中就已经接触到的最基本的不等式,在数分高代中依然屡见不鲜!

\begin{proposition}[Cauchy-Schwarz不等式]
	对实数$a_1,a_2,\dots,a_n$和$b_1,b_2,\dots,b_n$有不等式:
	\begin{equation}
		\left|\displaystyle\sum_{i=1}^{n}a_ib_i\right|\leq\sqrt{\displaystyle\sum_{i=1}^{n}{a_i}^2}\sqrt{\displaystyle\sum_{i=1}^{n}{b_i}^2}
	\end{equation}
\end{proposition}

\begin{corollary}[柯西不等式积分形式]
	假设函数 $f(x)$ 和 $g(x)$ 在闭合区间 $[a, b]$ 上连续,则有不等式:
	\begin{equation}
		\left(\int_a^b f(x)g(x) \, dx\right)^2 \leq \int_a^b f(x)^2 \, dx \cdot \int_a^b g(x)^2 \, dx,
	\end{equation}
	成立。其中$f(x)$ 和 $g(x)$ 是定义在 $[a, b]$ 上的可积函数。
\end{corollary}

两种形式的柯西不等式,分别对应求和和积分两种情况——这也让我们有了初步的对求和和积分之间建立桥梁的意识,二者在做题时在不同情况下有非常好的效果!

\begin{proposition}
	如果$0<x<\frac{\pi}{2}$,则有不等式:
	\begin{equation}
		\sin{x}<x<\tan{x}.
	\end{equation}
		$\forall x\geq 0$有不等式:
	\begin{equation}
		\sin{x}\leq x.
	\end{equation}
	$\forall x\in R$有不等式:
	\begin{equation}
		|\sin{x}|\leq |x|.
	\end{equation}
\end{proposition}


