\section{两个求分式数列极限的杀招}

\begin{proposition}[Cauchy命题]
	设数列$\{x_n\}$收敛于$a$,则有:
	\begin{equation}
		\displaystyle\lim_{n\to\infty}\dfrac{x_1+x_2+\dots+x_n}{n}=a.
	\end{equation}
	成立,其中$a$可以是有限数,也可以是$+\infty$或$-\infty$,但不能是$\infty$.
\end{proposition}

\begin{proposition}[Stolz定理]
	设$\{b_n\}$是严格单调递增且趋于$+\infty$的数列,如果:
	\[\displaystyle\lim_{n\to\infty}\dfrac{a_n-a_{n-1}}{b_n-b_{n-1}}=A,\]
	则有:
	\begin{equation}
		\displaystyle\lim_{n\to\infty}\dfrac{a_n}{b_n}=A.
	\end{equation}
	成立,其中$A$可以是有限数,也可以是$+\infty$或$-\infty$,但不能是$\infty$.
\end{proposition}

这两个结论相当的重要,都没有在课本上出现,但却是求数列极限的杀招,尤其是针对分式型数列。