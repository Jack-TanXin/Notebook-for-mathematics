\section{连续函数基础}

\begin{definition}
	如果有$\displaystyle\lim_{x\to a}f(x)=f(a)$成立,则称$f(x)$在点$x=a$处连续;如果函数$f(x)$在区间$I$上的每个点上连续,则称$f(x)$在区间$I$上连续.
\end{definition}

如果一个函数$f(x)$在点$x_0$处不连续,那么则称$x=x_0$是$f(x)$的间断点,对于间断点类别的判定有下面的结论:

\begin{proposition}[间断点的判定]
	\begin{enumerate}
		\item 若$\displaystyle\lim_{x\to x_0}f(x)=A\neq f(x_0)$,则称$x_0$为$f(x)$的可去间断点.
		\item 若$\displaystyle\lim_{x\to x_0^+}f(x)\neq\displaystyle\lim_{x\to x_0^-}f(x)$,则称$x_0$为$f(x)$的跳跃间断点.
		\item 若$f(x)$在$x_0$处的左极限或右极限不存在,则称$x_0$为$f(x)$的第二类间断点.
	\end{enumerate}
	PS:可去间断点和跳跃间断点统称为第一类间断点.
\end{proposition}