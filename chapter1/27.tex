\section{反函数的秘密}

\begin{definition}[反函数的定义]
	对于给定的函数 $f$,如果存在另一个函数 $g$,使得对于 $f$ 的定义域内的每个元素 $x$,都有:
	\begin{equation}
		g(f(x)) = x,
	\end{equation} 
	并且对于 $g$ 的定义域内的每个元素 $y$,都有:
	\begin{equation}
		f(g(y)) = y,
	\end{equation}
	则函数 $g$ 称为函数 $f$ 的反函数。
\end{definition}

\begin{definition}[单射、满射与双射]
	对于定义在$D$上的值域为$I$的函数$f(x)$:
	\begin{enumerate}
		\item $\forall x_1,x_2\in D$,若$f(x_1)=f(x_2)$则有$x_1=x_2$,则称$f$为单射.
		\item $\forall y\in I$,都存在$x\in D$,使得 $f(x) = y$,则称函数 $f$ 为满射.
		\item 既是单射又是满射的映射是双射.
	\end{enumerate}
	
\end{definition}

\begin{definition}[单调函数的定义]
	设$f(x)$为定义在$D$上的函数,若对$\forall x_1,x_2\in D$,当$x_1<x_2$时总有:
	\begin{enumerate}
		\item $f(x_1)\leq f(x_2)$,则称$f$为$D$上的增函数.特别地,若有不等式$f(x_1)< f(x_2)$成立,则称$f$为$D$上的严格增函数.
		\item $f(x_1)\geq f(x_2)$,则称$f$为$D$上的减函数.特别地,若有不等式$f(x_1)> f(x_2)$成立,则称$f$为$D$上的严格减函数.
	\end{enumerate}
	增函数和减函数统称为单调函数,严格增函数和严格减函数统称为严格单调函数.
\end{definition}

\begin{proposition}
	\begin{enumerate}
		\item 若函数$f$为双射,则$f$一定有反函数.反之不成立.
		\item 若函数$f$在$D$上严格单调递增(递减),则$f$在$D$上一定有反函数.反之不成立.
	\end{enumerate}
\end{proposition}
