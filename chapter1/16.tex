\section{闭区间上连续函数的性质}

闭区间上的连续函数有很多很好的性质,本部分逐个介绍:

\begin{theorem}[最大值最小值定理]
	设$f$为闭区间$[a,b]$上的连续函数,则$f(x)$在$[a,b]$上有最大值、最小值.
\end{theorem}

\begin{corollary}[有界性定理]
	设$f$为闭区间$[a,b]$上的连续函数,则$f(x)$在$[a,b]$上有界.
\end{corollary}
	
	
\begin{theorem}[介值性定理]
	设$f$为闭区间$[a,b]$上的连续函数,且有$f(a)\neq f(b)$,若$\mu$为介于$f(a)$与$f(b)$的常数,则至少存在一点$x_0\in(a,b)$使得:
	\begin{equation}
		f(x_0)=\mu.
	\end{equation}
\end{theorem}

\begin{corollary}[根的存在性定理]
	设$f$为闭区间$[a,b]$上的连续函数,且$f(a)$与$f(b)$异号($f(a)f(b)<0$),则至少存在一点$x_0\in (a,b)$使得:
	\begin{equation}
		f(x_0)=0.
	\end{equation}
	即$f(x)$在$(a,b)$至少存在一个根.
\end{corollary}