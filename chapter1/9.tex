\section{数列的上下极限}

数列的上下极限有两种定义方式,二者是等价的,这里逐个介绍:

\begin{definition}[极限点]
	数列的极限点是数列的收敛子列的极限。特别地,若存在正(负)无穷大量的子列,亦将$+\infty(-\infty)$作为该数列的极限点.
\end{definition}

\begin{definition}[第一定义]
	数列的上极限是该数列最大的极限点,记为$\displaystyle\varlimsup_{n\to\infty}x_n$;数列的下极限是该数列最小的极限点,记为$\displaystyle\varliminf_{n\to\infty}x_n$.
\end{definition}

从上下极限的第一定义可以看出:

\begin{enumerate}
	\item 如果一个数列收敛,则其上下极限都存在且都等于极限值.
	\item 对于$\forall\{x_n\}$有$\displaystyle\varlimsup_{n\to\infty}x_n\geq\displaystyle\varliminf_{n\to\infty}x_n.$
\end{enumerate}

\begin{definition}[第二定义]
	每个数列都有上下极限,且有:
	\begin{equation}
		\displaystyle\varlimsup_{n\to\infty}x_n=\displaystyle\lim_{n\to\infty}\sup_{k\geq n}\{x_k\}.
	\end{equation}
	\begin{equation}
		\displaystyle\varliminf_{n\to\infty}x_n=\displaystyle\lim_{n\to\infty}\inf_{k\geq n}\{x_k\}.
	\end{equation}
\end{definition}