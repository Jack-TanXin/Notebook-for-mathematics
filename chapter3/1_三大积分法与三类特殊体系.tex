\section{三大积分法与三类特殊体系}

\subsection{三大积分法}

\begin{theorem}[第一换元积分法,凑微分法]
	设函数$f$在区间$I$上有定义,$\varphi(t)$在区间$J$上可导,且$\varphi(J)\in I$.如果不定积分$\int f(x)\textbf{d}x = F(x)+C$在$I$上存在,则不定积分$\int f(\varphi(t))\varphi'(t)\textbf{d}t$在$J$上也存在,且有:
	\begin{equation}
		\int f(\varphi(t))\varphi'(t)\textbf{d}t = F(\varphi(t))+C.
	\end{equation}
\end{theorem}

\begin{theorem}[第二换元积分法,代入换元法]
	设函数$f(x)$在区间$I$上有定义,$\varphi(t)$在区间$J$上可导,$\varphi(J)=I$,且$x=\varphi(t)$在区间$J$上存在反函数$t = \varphi^{-1}(x),x\in I$.如果不定积分$\int f(x)\textbf{d}x$在区间$I$上存在,则当不定积分$\int f(\varphi(t))\varphi'(t)\textbf{d}t=G(t)+C$在$J$上存在时,在$I$上有:
	\begin{equation}
		\int f(x)\textbf{d}x = G(\varphi^{-1}(x))+C.
	\end{equation}
\end{theorem}

\begin{theorem}[分部积分法]
	若$u(x)$和$v(x)$可导,不定积分$\int u'(x)v'(x)\textbf{d}x$存在,则$\int u(x)v'(x)\textbf{d}x$也存在,并有:
	\begin{equation}
		\int u(x)v'(x)\textbf{d}x = u(x)v(x)-\int u'(x)v(x)\textbf{d}x.
	\end{equation}
\end{theorem}

\subsection{有理函数和可化为有理函数的不定积分}

